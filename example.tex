\section{Definities}

\begin{frame}
\frametitle{Definities}
\begin{block}{}
Een verzameling $S$, uitgerust met een afbeelding $\cdot : S \times S \to S$ noemt men een \textbf{semigroep} als de afbeelding $(\cdot)$ associatief is. Of nog $(ab)c = a(bc)$ voor alle $a,b,c \in S$.
\end{block}

\begin{block}{}
$x$ is een \textbf{neutraal element} als $\forall s \in S: sx = s = xs$. 

$x$ is een \textbf{nulelement} als $\forall s \in S: sx = x = xs$. 
\end{block}

\end{frame}

\begin{frame}
\frametitle{$M(n,\F)$}

\begin{block}{}
$M(n,\F) = \{ f : \F^n \to \F^n \; | \; f \text{ is lineair} \}$ is een semigroep met nulelement.
\begin{enumerate}
\item Samenstelling van lineaire functies is lineair.
\item Samenstelling is associatief.
\item We hebben een eenheid op $M(n,\F)$ (de identieke afbeelding).
\item We hebben een nulelement op $M(n,\F)$ (de nul afbeelding).
\end{enumerate}

\end{block}

\end{frame}


%\begin{frame}
%\frametitle{toevoegen van eenheid}
%
%\begin{block}{}
%stel $S^1 = S \cup \{1\}$, breid de bewerking op $S$ uit:
%$$ \cdot : S^1 \times S^1 \to S' : (a,b) \to \left\{ 
%\begin{array}{cl} 
%ab & a,b \in S \\
%a & a \in S \\
%b & b \in S \\
%1 & 
%\end{array}
%\right. $$
%\end{block}
%
%\begin{block}{}
%Als $S$ al een neutraal element heeft stellen we $S^1 = S$
%\end{block}
%
%\end{frame}


\begin{frame}
\frametitle{Nilpotente deelsemigroep}

\begin{block}{}
Zij $S$ een semigroep met nulelement $0$. $S' \subset S$ noemen we \textbf{nilpotent} als 
$$\exists k \in \N : S'^k = \{ a_1...a_k \; | \; a_1, ..., a_k \in S' \} = \{0 \} $$
De nilpotentie klasse $nd(S)$ is de kleinste $k \in \N$ zodat $S'^k = \{ 0\}$
\end{block}

\begin{block}{}
We noemen een element $s \in S$ nilpotent als $s^k = 0$ voor een $k \in \N$.
\end{block}

\end{frame}

\begin{frame}
\frametitle{Vlaggen}

\begin{block}{}
Zij $V$ een eindig dimensionale vectorruimte. Een \textbf{vlag} $\mathcal{F}$ is een rij deel vectorruimten $V_i$ zodat 
$$\{ 0 \} = V_0 \subsetneq V_1 \subsetneq ... \subsetneq V_k = V.$$ 
We noemen $l(\mathcal{F}) = k$ de lengte van de vlag $\mathcal{F}$.

\end{block}

\begin{block}{}
Een $\mathcal{F}$-basis is een basis $E = (e_1, ..., e_n)$ zodat $ \vect \{e_1, ..., e_{\dim(v_i)} \} = V_i$.
\end{block}

\end{frame}


\section{Link tussen vlaggen en nilpotente semigroepen}

\begin{frame}
\frametitle{Definitie $\varphi(\mathcal{F})$ en $\psi(S)$}

\begin{block}{}
Associ\"eer met een vlag $\mathcal{F}$ de nilpotente deelsemigroep
$$\varphi (\mathcal{F}) = \{ a \in M(n,\F) \; | \; \forall i :  a(V_i) \subset V_{i-1} \}$$
van $M(n,\F)$.
\end{block}

\begin{block}{}
Associ\"eer met een nilpotente deelsemigroep $S$ de vlag $\psi (S) $ gedefinieerd door:
$$ \{0 \} \subsetneq \langle S^{r-1} (\F^n) \rangle \subsetneq \dots \subsetneq \langle S(\F^n)\rangle \subsetneq \F^n$$
\end{block}

\begin{block}{Eigenschap}
$\varphi(\mathcal{F})$ is een nilpotente deelsemigroep, en $\psi(S)$ is een vlag in $\F^n$.
\end{block}


\end{frame}


\begin{frame}
\frametitle{$r$-maximale nilpotente deelsemigroepen}

\begin{block}{}
Beschouw
$$N_r(S) = \{ S' \subset S \; | \; S' \text{ nilpotente semigroup, } nd(S') = r\}.$$
We noemen $S'$ $r$-maximaal als $S'$ maximaal is in $N_r(S)$ voor de inclusie relatie. 
\end{block}

\end{frame}


\begin{frame}
\frametitle{ $ $ }

\begin{block}{Eigenschap}
Zij $\mathcal{F}$ een vlag in $M(n,\F)$ van lengte $k$. Dan is $\varphi (\mathcal{F})$ een $k$-maximale nilpotente deelsemigroep, en voor elke $k$-maximale nilpotente deelsemigroep $S$ bestaat er een vlag $\mathcal{F'}$ zodat $\varphi (\mathcal{F}') = S$.
\end{block}

\begin{block}{}
Dus de $k$-maximale nilpotente deelsemigroepen en de vlaggen van lengte $k$ staan in bijectief verband.
\end{block}

\end{frame}


\section{Link tussen signature en isomorfisme}

\begin{frame}
\frametitle{Signature}

\begin{block}{}
Voor een vlag $\mathcal{F}$ van lengte $r$ stellen we 
$$\sig(\mathcal{F}) = (d_1, ..., d_r)$$
waar $d_i = \dim(V_i/V_{i-1})$.
\end{block}

\begin{block}{}
Voor een $r$-maximale nilpotente deelsemigroep $S$ stellen we $\sig(S) = \sig(\psi(S))$.
\end{block}

\end{frame}


\begin{frame}
\frametitle{Voor $r = 2$}

\begin{block}{}
Als $\F$ eindig is en $S = \varphi(\mathcal{F})$, $T = \varphi(\mathcal{F'})$ $2$-maximale nilpotente deelsemigroepen zijn, dan zijn $T$ en $S$ isomorf als en slechts als 
$$\{ \dim (V_1), \dim (V_1/V_2) \} = \{ \dim (W_1), \dim (W_1/W_2) \}.$$
\end{block}

\begin{block}{}
Als $\F$ oneindig is, dan zijn alle $2$-maximale nilpotente semigroepen isomorf.
\end{block}

\end{frame}

\begin{frame}
\frametitle{Hoofdstelling}
\begin{block}{Eigenschap}
zij $S$ en $T$ $r$-maximaal nilpotente deelsemigroepen, met $r\geq 3$.
\begin{enumerate}
\item Als $\F$ eindig is, dan 
$$\varphi(\mathcal{F}) \cong \varphi(\mathcal{F}') \Leftrightarrow \sig (\mathcal{F}) = \sig (\mathcal{F}').$$
\item Als $\F$ oneindig is en $\sig (\mathcal{F}) = (k,1,l)$, dan 
$$\varphi(\mathcal{F}) \cong \varphi(\mathcal{F}') \Leftrightarrow \sig(\mathcal{F}') = (k',1,l').$$
\item Als $\F$ oneindig is, en $\sig(\mathcal{F)} \neq (k,1,l)$, dan 
$$\varphi(\mathcal{F}) \cong \varphi(\mathcal{F}') \Leftrightarrow \sig(\mathcal{F}) = \sig(\mathcal{F}').$$
\end{enumerate}

\end{block}

\end{frame}


\begin{frame}
\frametitle{Voor $\sig(\mathcal{F}) = (k,1,l)$}

\begin{block}{Eigenschap}
Zij $\F$ een oneindig veld. Zij $S$ en $T$ $3$-maximale nilpotente deelsemigroepen van $M(n,\F)$ met signature $(k,1,l)$ en $(k',1,l')$ met $k,l,k',l' > 1$. Dan zijn $S$ en $T$ isomorf.
\end{block}

\begin{block}{}
Maak gebruik van de vorm van $A \in \varphi(\mathcal{F})$ om rechtstreeks een isomorfisme te definieren.
\end{block}

\end{frame}

\begin{frame}
\frametitle{ $ $ }

\begin{block}{}
probeer de $d_i$ in $\sig(\mathcal{F})$ te caracterizeren door iets dat bewaard blijft onder isomorfisme van semigroepen. 
\end{block}

\begin{block}{}
Zij $T$ een nilpotente deelsemigroep, en $A \in T$.

$A$ is onontbindbaar als $\forall B,C \in T: A \neq BC$
\end{block}

\end{frame}


\begin{frame}
\frametitle{Relaties $\prec$ en $\ll$}

\begin{block}{}
Zij $X \subseteq M(n,\F)\setminus \{0\} $. 

Stel voor $A,B \in X$, $A\prec_X B$ als 
$$\forall C \in X : AC = 0 \Rightarrow BC = 0.$$

Stel voor $A,B \in X$, $A \ll_X B $ als 
$$\forall C \in X : CA = 0 \Rightarrow CB = 0.$$

\end{block}

\begin{block}{}
$\prec$ en $\ll$ zijn preordes op $X$, ze zijn reflexief en transitief.
\end{block}


\end{frame}

\begin{frame}
\frametitle{ $ $ }

\begin{block}{}
Zij $\leq$ een partiele orde op $X$, $Y \subseteq X$. $Y$ is een ketting als 
$$ \forall x,y \in Y : x \leq y \vee y \leq x .$$
\end{block}

\begin{block}{}
Stel 
$$m_y = \max \{ \card (Y) \; | \; Y \text{ is een ketting op } X, \forall x \in Y: y \leq x \}$$ 
de diepte van $y$
\end{block}

\begin{block}{}
Stel 
$$M_{i}^{\leq} = \{ y \in X \; | \; m_y = i\}$$
\end{block}

\end{frame}



\begin{frame}
\frametitle{Super rank}

\begin{block}{}
Definieer verzamelingen
$$K_{1,0} = M_{1}^{\prec} \cap M_{0}^{\ll}$$
$$K_{0,1} = M_{0}^{\prec} \cap M_{1}^{\ll}$$
$$K_{1,1} = \{ A \in M_{1}^{\prec} \cap M_{1}^{\ll} \; | \; TAT \neq 0  \}$$
\end{block}

\begin{block}{}
Zij $A$ onontbindbaar voor $T$, als $A \in K_{1,0} \cup K_{0,1} \cup K_{1,1}$, dan stellen we $\supR (A) = 1$.
\end{block}


\end{frame}

\begin{frame}
\frametitle{Super rank}

\begin{block}{}
$$K_{2,0} = M_{2}^{\prec} \cap M_{0}^{\ll}$$
$$K_{0,2} = M_{0}^{\prec} \cap M_{2}^{\ll}$$
$$K_{2,1} = \{ A \in M_{2}^{\prec} \cap M_{1}^{\ll} \; | \; TAT \neq 0  \}$$
$$K_{1,2} = \{ A \in M_{1}^{\prec} \cap M_{2}^{\ll} \; | \; TAT \neq 0  \}$$
$$K_{2,2} = \{ A \in M_{2}^{\prec} \cap M_{2}^{\ll} \; | \; TAT \neq 0; \forall B \in M(n,\F)$$
$$ \text{ met } \supR(B) = 1 : TAT \neq TBT \}$$
\end{block}

\begin{block}{}
Zij $A$ onontbindbaar voor $T$, Als $A \in K_{2,0} \cup K_{0,2} \cup K_{1,2} \cup K_{2,1} \cup K_{2,2}$. Dan stellen we $\supR(A) = 2$.
\end{block}

\end{frame}


\begin{frame}
\frametitle{Isorfisme bewaard $\supR$}

\begin{block}{Eigenschap}
Zij $T_1$ en $T_2$ twee deelsemigroepen van $M(n,\F)$, en $\phi :T_1 \to T_2$ een isomorfisme. Zij $A \in T_1$ een onontbindbaar element zodat $\supR(A) = 1$, dan is ook $\phi(A)$ onontbindbaar en $\supR(\phi(A)) = 1$.
\end{block}

\end{frame}


\begin{frame}
\frametitle{Uitbreiding $\supR$}

\begin{block}{}
Zij $A \in T\setminus \{0 \}$ een ontbindbaar element,
\end{block}


\begin{block}{}
$$\supR(A) = 1$$
$$\Leftrightarrow$$
$$A = A_1...A_m, A_i \text{ onontbindbaar, } \exists i : \supR(A_i) = 1$$
\end{block}

\begin{block}{}
$$\supR(A) = 2$$
$$\Leftrightarrow$$
$$\supR(A) \neq 1, A = A_1...A_m, A_i \text{ onontbindbaar, } $$
$$\exists i : \supR(A_i) = 2$$
\end{block}

\end{frame}


\begin{frame}
\frametitle{ $ $ }

\begin{block}{Eigenschap}
Zij $T_1$ en $T_2$ twee isomorfe $r$-maximale nilpotente deelsemigroepen van $M(n,\F)$, met $r \geq 3$. Schrijf $\sig(T_1) = (i_1, ..., i_r)$ en $\sig(T_2) = (j_1, ..., j_r)$. Dan is $i_l = j_l$ voor $2 \leq l \leq r-1$.
\end{block}

\begin{block}{}
Door gebruik te maken van cardinaliteit van verzamelingen gedefini\"eerd aan de hand van $\supR$.
\end{block}

\end{frame}


\section{Overige gevallen}

\begin{frame}
\frametitle{Overige gevallen}

\begin{block}{}
$S$ en $T$ twee isomorfe $r$-maximale nilpotente deelsemigroepen, met $\sig(S) = (i_1, ...,i_r)$.

\begin{enumerate}
\item $i_1 \geq 1, i_r \geq 2, \sum_{r-1}^{s=2} i_s \geq 1 $,
\item $i_1 = 1, \sum_{r-1}^{s=2} i_s \geq 2 $,
\item $r=3, i_1 = 1, i_3 = 1 $,
\item $r=3, i_1 \geq 2, i_2 = 1, i_3 \geq 2 $ indien $\F$ eindig.
\end{enumerate}

\end{block}

\begin{block}{}
Blijft te tonen dat in deze gevallen ook $i_1 = j_1$ en $i_r = j_r$. 
\end{block}

\end{frame}


\section{Einde}

\begin{frame}
\frametitle{{\color{blurred}Overige gevallen}}

\begin{block}{}
{\color{lightgray}$S$ en $T$ twee isomorfe $r$-maximale nilpotente deelsemigroepen, met $\sig(S) = (i_1, ...,i_r)$.}

\begin{enumerate}[-]
\item {\color{lightgray}$i_1 \geq 1, i_r \geq 2, \sum_{r-1}^{s=2} i_s \geq 1 $,}
\item {\color{lightgray}$i_1 = 1, \sum_{r-1}^{s=2} i_s \geq 2 $,}
\item {\color{lightgray}$r=3, i_1 = 1, i_3 = 1 $,}
\item {\color{lightgray}$r=3, i_1 \geq 2, i_2 = 1, i_3 \geq 2 $ indien $\F$ eindig.}
\end{enumerate}

\end{block}

\begin{block}{}
{\color{lightgray}Blijft te tonen dat in deze gevallen ook $i_1 = j_1$ en $i_r = j_r$.}
\end{block}

\end{frame}


\begin{frame}
\frametitle{{\color{blurred}Overige gevallen}}

\begin{block}{}
{\color{lightgray}$S$ en $T$ twee isomorfe $r$-maximale nilpotente deelsemigroepen, met $\sig(S) = (i_1, ...,i_r)$.}

\begin{enumerate}[-]
\item {\color{lightgray}$i_1 \geq 1, i_r \geq 2, \sum_{r-1}^{s=2} i_s \geq 1 $,}
\item {\color{lightgray}$i_1 = 1, \sum_{r-1}^{s=2} i_s \geq 2 $,}
\item {\color{lightgray}$r=3, i_1 = 1, i_3 = 1 $,}
\item {\color{lightgray}$r=3, i_1 \geq 2, i_2 = 1, i_3 \geq 2 $ indien $\F$ eindig.}
\end{enumerate}

\end{block}

\begin{block}{}
{\color{lightgray}Blijft te tonen dat in deze gevallen ook $i_1 = j_1$ en $i_r = j_r$.}
\end{block}

\end{frame}
