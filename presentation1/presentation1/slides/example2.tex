\begin{frame}[fragile]
\frametitle{Example}
	\begin{block}{Plugin}
        \begin{lstlisting}[language=Python]

def _activate_in_arta(subscription):
    """
    ...
    """
    if not gargoyle.is_active('arta:provisioning:activations'):
        return False

    # Create backendcli and contact in Arta if necessary
    arta_contact_already_exists = False
    try:
        cli = subscription.backendcli
        if cli.contactid:
            arta_contact_already_exists = True
    except BackendCLI.DoesNotExist:
        cli = BackendCLI.objects.create(subscription=subscription)
    if not arta_contact_already_exists:
        if not create_arta_contact(subscription):
            return False
        # Needs to be 'refreshed' to get the correct value
        cli = subscription.backendcli

    # Get the SN that's provisioned on the SIM
    if cli.sn.strip() == '':
        cli.sn = get_sn_for_sim(subscription.mobilenumber.active_sim)
        cli.save()

    # Push the CLI to the next state if necessary
    state = State.objects.default(cli)
    if state is None:
        return False
    elif state.current_is['CLI_SIM_ACTIVE']:
        result = True
    elif state.current_is['NOT_IN_ARTILIUM']:
        result = state.action(None, 'attach_cli_to_user')
    elif state.in_group['can_resume']:
        result = state.action(None, state.get_action_names()[0])
    else:
        return False

    # Set all warning SMS as SENT
    for f in Flag.objects.filter(name__startswith='ALERT', name__endswith='SENT'):
        f.add_subscription(subscription)

    if result:
        if not subscription.has_number_porting:
            Status.objects.sim_activated(subscription)
        else:
            Status.objects.number_porting_requested(subscription)

        # mark sim as active
        act = Activation.objects.get(subscription=subscription)
        act.activated_on = datetime.now()
        act.save()

        # Put all topups on hold back to paymentdone
        subscription.topup_set.get_on_hold().update(status=TOPUP_STATUS_PAYMENT_DONE)
    else:
        from mvne.activation.utils import send_activation_failure_email
        send_activation_failure_email(subscription)
        return False

    return True
        \end{lstlisting}
	\end{block}
\end{frame}

\begin{frame}
    \frametitle{Issues}
    \begin{block}
        \begin{itemize}
            \item Function is long (> 15 lines)
            \item Function intent is clear by name, but details of function requires effort understand
            \item Code is seperated in logical chunks but no real abstraction mechanism is used
        \end{itemize}
    \end{block}

\end{frame}

\begin{frame}
    \frametitle{My beef with comments}
    \begin{block}
        \begin{itemize}
            \item Will get outdated when code changes
            \item Gives wrong/incomplete information
            \item Often used as an alternative readable code
            \item Less tought given to good naming
        \end{itemize}
    \end{block}
\end{frame}

\begin{frame}
    \frametitle{Example}
	\begin{block}{Plugin}
        \begin{lstlisting}

def _activate_in_arta(subscription):
    """
    ...
    """
    if not gargoyle.is_active('arta:provisioning:activations'):
        return False

    cli = get_or_create_backend_cli()

    set_sn_on_sim(subscription, cli)
    next_state = push_cli_to_next_state(cli)
    notify_sim_sent(subscription)
    activate_subscription(subscription)
        \end{lstlisting}
	\end{block}
\end{frame}

\begin{frame}
    \frametitle{Benefits}
    \begin{block}
        \begin{itemize}
            \item Content of function is all at the same level of detail.
            \item Easier to understand the function in question. More details can be found for each seperate step
            \item Forces more thought through names (no one wants superlong names, but intent needs to be clear)
            \item Incentivizes functional style (wich is clearer IMO)
            \item Decentivizes statechange in between logical chunks
        \end{itemize}
    \end{block}
\end{frame}
